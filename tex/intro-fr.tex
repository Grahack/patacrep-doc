%*******************************************************************************
%*******************************************************************************
\addentry{toc}{section}{\textbf{Introduction}}
\chapter*{Introduction}
\minitoc
\label{chap:introduction}
\chaptermark{Introduction}
%*******************************************************************************
%*******************************************************************************

Dans ce document, nous expliquons comment utiliser et tirer pleinement
partie des outils et ressources proposés sur
\url{http://www.patacrep.com} pour réaliser des recueils de
chansons. Tout d'abord, une présentation rapide des deux différents
projets, le \songbook et son interface, le \client.

\paragraph{Songbook}
Le \songbook est un projet libre et gratuit pour la production de
recueils de tablatures. Il est rendu possible par le projet libre
\songs, développé par \href{http://www.utdallas.edu/~hamlen}{Kevin
  W. Hamlem} et permettant d'écrire des recueils de tablatures en
\latex. Le \songbook offre également la possibilité d'intégrer des
partitions ou des extraits de partitions générés par
\href{http://lilypond.org}{Lilypond}.

Le \songbook correspond à la fois~:
\begin{itemize}
\item à un ensemble d'outils (scripts, makefile) permettant de
  simplifier et d'automatiser la création d'un recueil~;
\item à une base de données des tablatures, réalisée collaborativement
  avec la communauté d'utilisateurs de
  \href{http://www.patacrep.com}{Patacrep!}
\end{itemize}

Vous pouvez par exemple consulter le fichier
\href{http://www.patacrep.com/data/documents/songbook.pdf}{\file{songbook.pdf}}
comprenant l'ensemble des chansons pour vous faire une idée du
résultat.

\paragraph{Songbook-client}
Il s'agit d'une application graphique servant d'interface au
\songbook. L'application permet de sélectionner en un clic les chansons
à faire apparaître dans un recueil et fournit également un éditeur
intégré pour modifier ou ajouter une nouvelle tablature.

\paragraph{Licences}
Tout le code est distribué sous licence
\href{http://www.gnu.org/licenses/gpl.html}{GPLv2}. Tous les documents
et autres resources sont distribués sous une licence
\href{http://creativecommons.org/}{Creative Commons CC-By-Sa}.
