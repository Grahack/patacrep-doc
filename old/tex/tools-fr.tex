%*******************************************************************************
%*******************************************************************************
\chapter{Outils externes}
\setcounter{chapter}{3}
\label{chap:outils-externes}
\minitoc
%*******************************************************************************
%*******************************************************************************

%*******************************************************************************
\section{\LaTeX{}}
%*******************************************************************************

\LaTeX{} est un logiciel de traitement de texte qui va générer un
document pdf depuis un fichier source (un simple fichier texte). Cela
permet à l'utilisateur de se concentrer sur le contenu plutôt que sur
la forme. Pour que \LaTeX{} puisse produire un rendu correct, la
rédaction d'un fichier source doit respecter certaines contraintes
basées sur l'utilisation de commandes.

Une commande commence par un antislash ($\backslash$) suivi du nom de
la commande. Ensuite viennent les arguments, s'il y en a, entre
accolades (obligatoires) ou crochets (optionnels).


%*******************************************************************************
\section{Git}
%*******************************************************************************

%-------------------------------------------------------------------------------
\subsection{Mise à jour du dépôt}
%-------------------------------------------------------------------------------

Pour maintenir à jour la version du songbook avec les dernières
chansons rajoutées, vous pouvez télécharger la dernière version mise
en ligne de l'archive tar.gz ou, si vous avez utilisé Git pour
récupérer les sources~:
\begin{unix}
  git pull
\end{unix} 

%commandes de base
%git clone add/rm commit push/pull status

%récupérer un dépôt

%mettre à jour

%modifier

%command avancées
%git branch amend checkout rebase
%github

%-------------------------------------------------------------------------------
\subsection{Création d'un patch}
%-------------------------------------------------------------------------------

Une bonne façon de nous faire remonter vos corrections/modifications
est de nous les envoyer sous forme de patch. Techniquement, un patch
est un simple fichier qui indique les opérations (ajout/suppression)
qui ont été opérées sur un fichier texte. Un exemple tout simple a
l'aspect suivant~:

\begin{code}
index 3fcce15..b4edcc1 100644
--- a/songs/Cat_Stevens/The_wind.sg
+++ b/songs/Cat_Stevens/The_wind.sg
@@ -7,7 +7,7 @@

-\[Do] I listen to the \[Fa]wind,
+\[Ré] I listen to the \[Sol]wind,
\end{code}

Vous voyez rapidement qu'après les premières lignes qui servent à
identifier le fichier sur lequel on a travaillé, on a remplacé~:
\begin{code}
-\[Do] I listen to the \[Fa]wind,
\end{code}
par
\begin{code}
+\[Ré] I listen to the \[Sol]wind,
\end{code}

Concrètement, si vous avez suivi la procédure d'installation, vous
devriez avoir un répertoire \directory{\$HOME/songbook/} contenant toutes
les sources du carnet de chant. Maintenant, vous avez repéré une
erreur dans la chanson \file{songs/artiste/chanson.sg}. Ouvrez le
fichier correspondant avec votre éditeur de texte préféré, faîtes
votre correction, enregistrez, fermez. Puis, dans un terminal~:

\begin{unix}
  git diff -u songs/artiste/chanson.sg > patch
\end{unix} 

Cela va vous créer un nouveau fichier \file{patch} qui contiendra
l'ensemble des modifications que vous avez apportées à votre
chanson. Plus qu'à nous l'envoyer !


%-------------------------------------------------------------------------------
\subsection{Forker un projet}
%-------------------------------------------------------------------------------

Si vous prévoyez de faire beaucoup de modifications ou de rejoindre
activement le développement du songbook, le système de patch n'est pas
forcément adapté et il vaut mieux dans ce cas partir sur votre propre
version du songbook. La meilleure solution consiste à faire un
\emph{fork} du projet Git et de l'héberger sur le web si vous comptez
le partager. Des solutions web gratuites permettent d'héberger de tels
projets. Par exemple~:

\begin{itemize}
\item \url{http://github.com}
\item \url{http://gitorious.org}
\end{itemize}

Il suffit alors de nous communiquer l'adresse de votre dépôt Git de
façon à ce que nous puissions récupérer vos changements.

%*******************************************************************************
\section{Lilypond}
%*******************************************************************************

La documentation du projet Lilypond est très
claire et se trouve sur le site \url{http://lilypond.org/}.
Voici néanmoins quelques concepts de base~:

\begin{itemize}
\item les lettres a, b, c, d, e, f, g représente les notes la, si, do,
  ré, mi, fa, sol~;
\item un chiffre derrière une lettre en indique la durée (2=blanche, 4=noire,
  8=croche) et un point après un chiffre désigne une note pointée~;
\item \emph{ais}, \emph{bes} désignent un \emph{la dièse} et un \emph{si bémol}
\item \emph{'} et \emph{,} servent à monter/descendre d'une octave.
\end{itemize}

Lilypond est généralement empaqueté pour les distributions
GNU/Linux. Pour des distributions basées Debian/Ubuntu, l'installation
se fait par~:

\begin{unix}
  sudo apt-get install lilypond
\end{unix}
