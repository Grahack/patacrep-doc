%*******************************************************************************
%*******************************************************************************
\addentry{toc}{section}{\textbf{Introduction}}
\chapter*{Introduction}
\minitoc
\label{chap:introduction}
\chaptermark{Introduction}
%*******************************************************************************
%*******************************************************************************

In this document, we detail how to use the tools that are available on
\url{http://www.patacrep.com} in order to produce customized
songbooks. Following is a brief insight of the two main projects:

\paragraph{Songbook}
This project corresponds to a set of scripts to build a songbook.  It
also contains the set of songs/tabs that have already been written.
You may download the whole pdf at
\url{http://www.patacrep.com/data/documents/songbook.pdf} as an
example.

\paragraph{Songbook-client} 
This application is a graphical interface to the songbook.

\paragraph{Technologies and upstream projects} 
The Patacrep songbook derives from the \emph{Songs LaTeX
  Package}\footnote{\url{http://songs.sourceforge.net/}}.

\begin{itemize}
\item \LaTeX~: songs are written according to a set of simple rules
  that allow to use \LaTeX{} for text rendering. As a result, a songbook
  is a document with high typographical quality ;
\item Lilypond~: free software to write high quality music sheets;
\item Makefile~: the production chain of a songbook is fully
  automated. Just type \emph{make} in a terminal.
\item Python~: several scripts, for example authors/titles indexes'
  generation, are written python.
\item C++/Qt4~: language/toolkit for the songbook-client application.
\end{itemize}

\paragraph{Licenses}
Software is distributed under the GPLv2
license\footnote{\url{http://www.gnu.org/licenses/gpl.html}}. All
documents and graphics are distributed under a free Creative Commons
license CC-By-Sa\footnote{\bysa~\url{http://creativecommons.org/}}.
